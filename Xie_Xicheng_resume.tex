\documentclass[11pt,a4paper,sans]{moderncv}
\moderncvstyle{classic}
\moderncvcolor{blue}

\usepackage[top=0.80in, bottom=0.80in, left=0.80in, right=0.80in]{geometry} 
\usepackage{multicol}

% Personal info
\name{Xicheng}{Xie}
\phone[mobile]{+1~(336)~448~8191}
\email{xicheng.xie@northwestern.edu}
\social[linkedin]{xx0224}
\address{}{Evanston, IL}{USA}

\begin{document}

\makecvtitle

% Education
\section{Education}
\cventry{09/2024 -- Present}{Ph.D. in Biostatistics}{Northwestern University, Feinberg School of Medicine}{Evanston, IL}{}{}
\cventry{09/2022 -- 05/2024}{M.S. in Biostatistics}{Columbia University, Mailman School of Public Health}{New York, NY}{}{}
\cventry{09/2018 -- 06/2022}{B.S. in Environmental Science}{Shanghai Jiao Tong University}{Shanghai, China}{}{}

% Skills
\section{Technical Skills}
\cvitem{Programming}{Python (TensorFlow, PyTorch, Keras, Scikit-learn, NumPy, Pandas, Matplotlib), SQL, R (glmnet, caret, dplyr, MASS, RShiny, Bioconductor), MATLAB, SAS, Shell/SSH}
\cvitem{Methodologies}{Machine Learning, Reinforcement Learning, Bayesian Statistics, Survival Analysis, Longitudinal Analysis, Time-Series Analysis, Missing Data Imputation, Prompt Tuning}
\cvitem{Tools \& Platforms}{MySQL, PyCharm, Jupyter Notebook, RStudio, Microsoft Azure, Git, AWS, Slurm}

% Experience
\section{Experience}

\cventry{02/2024 -- 09/2024}{Data Scientist}{IQVIA}{Shanghai, China}{}{
\begin{itemize}
    \item Built pipelines to extract influenza data from China CDC reports and Baidu Index data, creating a cleaned weekly time-series dataset (2012–2024).
    \item Engineered time-series features (lags, windows), performed dimensionality reduction, and applied model-based feature selection.
    \item Trained SARIMAX, LSTM, LightGBM, and XGBoost to predict ILI rates; SARIMAX improved short-term RMSE by 17\% and accuracy by 23\%.
\end{itemize}
}

\cventry{03/2023 -- 01/2024}{Data Analyst}{Columbia University}{New York, NY}{}{
\begin{itemize}
    \item Processed 40 years of US temperature data (12.7 GB) into a 500 MB longitudinal database on extreme heat events using Python and HPC.
    \item Developed Bayesian models to evaluate and forecast heat stress variations at the census-tract level nationwide, identifying vulnerable areas and groups.
    \item Built 2 interactive RShiny apps visualizing extreme heat and hurricane mortality, increasing climate report engagement by 12\%.
\end{itemize}
}

\cventry{01/2022 -- 09/2022}{Applied ML Engineer Intern}{Fresenius Medical Care}{Shanghai, China}{}{
\begin{itemize}
    \item Analyzed clinical and hemodynamic data from 600+ dialysis patients using Azure and Python to identify patient patterns.
    \item Built preprocessing and feature engineering pipelines to improve data quality for model training.
    \item Trained semi-supervised models (Random Forest, S3VM) to classify AV fistula conditions with 89.72\% accuracy, enhancing patient stratification and informing treatment protocols.
\end{itemize}
}

% Projects
\section{Projects}

\cvitem{Retail Pharmacy Forecasting}{
\begin{itemize}
    \item Led a time-series forecasting project using LSTM (TensorFlow) and Bayesian models (PyMC3) on 6 years of SKU-level sales data across 11 retail pharmacies.
    \item Engineered features from sequences of 1,095 days/SKU and achieved a 13.7\% improvement in forecasting accuracy over baseline.
    \item Applied Fuzzy C-Means clustering on sales patterns and seasonality, increasing inventory turnover of new products by 12\%.
\end{itemize}}

\cvitem{NYC Restaurant Recommendations}{
\begin{itemize}
    \item Scraped data from 8,000+ NYC restaurants on Yelp, merged with DOHMH inspection records, and cleaned 6,943 entries for analysis.
    \item Built an R Shiny App displaying interactive maps, inspection grades, and custom restaurant recommendations based on user location and preferences.
\end{itemize}}

\cvitem{Multi-Omics Cancer Subtyping}{
\begin{itemize}
    \item Developed a perturbation-based spectral clustering algorithm in R to subtype cancer patients using TCGA multi-omics data (mRNA, miRNA, DNA methylation).
    \item Identified subtypes for KIRC, GBM, and LAML with significant survival differences (Cox p-values $<$ 0.01); clustering validated across 8 datasets (RI: 0.91, ARI: 0.78).
\end{itemize}}


% Publications
\section{Publications}

\noindent Lynch, V.D., Sullivan, J.A., Flores, A.B., \textbf{Xie, X.}, Aggarwal, S., Nethery, R.C., Kioumourtzoglou, M.-A., Nigra, A.E., Parks, R.M. (2025). \textit{Large floods drive changes in cause-specific mortality in the United States}. \textit{Nature Medicine}, \textbf{31}, 663–671. \href{https://doi.org/10.1038/s41591-024-03022-3}{https://doi.org/10.1038/s41591-024-03022-3}

\medskip

\noindent Meltzer, G.Y., Anderson, G.B., \textbf{Xie, X.}, Casey, J.A., Schwartz, J., Bell, M.L., Van Horne, Y.O., Fox, J., Kioumourtzoglou, M.-A., Parks, R.M. (2025). \textit{Disruption to test scores after hurricanes in the United States}. \textit{Environmental Research: Health}, \textbf{3}(2), 025003. \href{https://doi.org/10.1088/2752-5309/adb32b}{https://doi.org/10.1088/2752-5309/adb32b}

\medskip

\noindent \textbf{Xie, X.C.}, Huang, W.W., Shen, G.Q., Yu, H., Wang, L.M. (2022). \textit{Selection and colorimetric application of ssDNA aptamers against metamitron based on magnetic bead-SELEX}. \textit{Analytical Methods}, \textbf{14}(31), 3021–3032. \href{https://doi.org/10.1039/d2ay00566b}{https://doi.org/10.1039/d2ay00566b}

\medskip

\noindent Yu, H., Pan, C.Q., Zhu, J.X., Shen, G.Q., Deng, Y., \textbf{Xie, X.C.}, Geng, X.Q., Wang, L.M. (2022). \textit{Selection and identification of a DNA aptamer for fluorescent detection of netilmicin}. \textit{Talanta}, \textbf{250}, 123708. \href{https://doi.org/10.1016/j.talanta.2022.123708}{https://doi.org/10.1016/j.talanta.2022.123708}

\medskip

\noindent \textbf{Xie, X.C.}, Li, L.Y., Wang, L.M., Pan, C.Q., Zhang, D.W., Shen, G.Q. (2021). \textit{Colourimetric detection of tebuconazole in aqueous solution based on an unmodified aptamer and the aggregation of gold nanoparticles}. \textit{Australian Journal of Chemistry}, \textbf{74}(12), 838–846. \href{https://doi.org/10.1071/ch21171}{https://doi.org/10.1071/ch21171}

\medskip

\noindent Shi, Y.Z., \textbf{Xie, X.C.}, Wang, L.M., Wang, L.Z., Li, L.Y., Yan, Z.Y., Shen, G.Q. (2022). \textit{Fluorescent assay for carbendazim determination using aptamer and SYBR Green I}. \textit{Australian Journal of Chemistry}, \textbf{75}(5), 345–352. \href{https://doi.org/10.1071/ch22001}{https://doi.org/10.1071/ch22001}






\end{document}
